\documentclass[12pt]{article}

% 导入必要的包
\usepackage[utf8]{inputenc} % 支持 UTF-8 编码
\usepackage{booktabs} % 用于美化表格
\usepackage{geometry} % 页面布局
\geometry{a4paper, margin=1in} % 页面设置

% 文档信息
\title{Paper Notebook}
\author{Sepine \\ Shanghai University}
\date{\today}

\begin{document}

\maketitle % 显示标题

\section{Introduction}
This document includes a table exported from Stata. Below is the table included from `09iv.tex`.
This document includes a table exported from Stata. Below is the table included from `09iv.tex`.

This document includes a table exported from Stata. Below is the table included from `09iv.tex`.
This document includes a table exported from Stata. Below is the table included from `09iv.tex`.
This document includes a table exported from Stata. Below is the table included from `09iv.tex`.

This document includes a table exported from Stata. Below is the table included from `09iv.tex`.
This document includes a table exported from Stata. Below is the table included from `09iv.tex`.

\section{Data}
\subsection{Source}
The data comes from lots of source including World Bank which I get the GDP data, IMF and OECD whcih I get the FDI data and exchange rate data.
Over them, I get the data of tea production from National Bureau of Statistics of China.
The import and export data whose HS code is matched as "09xxxx" which is also necessary is catched from EPS data platform.
\subsection{Summarize}
I keep the code matches "0901xx" and "0902xx", which is referred coffee and tea.
And here is the summarize statistics as follow table:

% {
\def\sym#1{\ifmmode^{#1}\else\(^{#1}\)\fi}
\begin{tabular}{l*{5}{cccccc}}
\toprule
                    &         Sum&        Mean&          SD&         Min&         Max&           N&         Sum&        Mean&          SD&         Min&         Max&           N&         Sum&        Mean&          SD&         Min&         Max&           N&         Sum&        Mean&          SD&         Min&         Max&           N&         Sum&        Mean&          SD&         Min&         Max&           N\\
\midrule
IM\_P                &            &            &            &            &            &            &            &            &            &            &            &            &            &            &            &            &            &            &            &            &            &            &            &            &      20,003&       32.06&       83.36&        0.49&     1876.00&         624\\
IM\_Q                &            &            &            &            &            &            &            &            &            &            &            &            &            &            &            &            &            &            &            &            &            &            &            &            &    11162785&    17226.52&     1.2e+05&        0.00&     1.9e+06&         648\\
limp                &            &            &            &            &            &            &            &            &            &            &            &            &            &            &            &            &            &            &            &            &            &            &            &            &       1,733&        2.78&        1.14&       -0.71&        7.54&         624\\
limq                &            &            &            &            &            &            &            &            &            &            &            &            &            &            &            &            &            &            &            &            &            &            &            &            &       3,308&        5.30&        3.33&        0.00&       14.46&         624\\
tea                 &            &            &            &            &            &            &            &            &            &            &            &            &            &            &            &            &            &            &            &            &            &            &            &            &      188729&      293.97&       33.56&      246.04&      354.11&         642\\
\bottomrule
\end{tabular}
}




\subsection{Elasticity}
As the response is expected to identified about the Covid-19, the elasticity is necessary for it.
Thus the calculation for elasticity is a problem for it, vast majority the elasticity is caused by "log-log" model' s parameters of $\beta_1$.

$$P = \beta_0 + \beta_1 \cdot Q + \epsilon$$

However, I find found that I could use a instrumental variable(IV) to get the elasticity more effective.

Differently with traditional formula, the iv function is more effective for it to solve endonal problem(Angrist and Krueger, 2001). The function as follow:

$$E = {Cov(P, IV) \over Cov(Q, IV)}$$

In order to find out a great IV for calculating the elasticity, there are two necessary conditions that the F-calue is over 10 at OLS and the p-value is less than 0.05.
Therefore the regression result is follow:

{
\def\sym#1{\ifmmode^{#1}\else\(^{#1}\)\fi}
\begin{tabular}{l*{4}{c}}
\toprule
            &\multicolumn{1}{c}{(1)}&\multicolumn{1}{c}{(2)}&\multicolumn{1}{c}{(3)}&\multicolumn{1}{c}{(4)}\\
            &\multicolumn{1}{c}{limp}&\multicolumn{1}{c}{limp}&\multicolumn{1}{c}{limq}&\multicolumn{1}{c}{limq}\\
\midrule
tea         &       0.019\sym{***}&       0.014\sym{***}&                     &                     \\
            &      (6.50)         &      (8.19)         &                     &                     \\
\addlinespace
limp        &                     &                     &      -4.552\sym{***}&      -2.062\sym{***}\\
            &                     &                     &     (-5.71)         &     (-8.30)         \\
\addlinespace
\_cons      &      -2.717\sym{**} &      -1.083\sym{*}  &      16.647\sym{***}&      12.312\sym{***}\\
            &     (-3.13)         &     (-2.18)         &      (7.25)         &     (15.93)         \\
\midrule
F           &      42.196         &      67.035         &                     &                     \\
p           &       0.000         &       0.000         &       0.000         &       0.000         \\
\bottomrule
\multicolumn{5}{l}{\footnotesize \textit{t} statistics in parentheses}\\
\multicolumn{5}{l}{\footnotesize \sym{*} \(p<0.05\), \sym{**} \(p<0.01\), \sym{***} \(p<0.001\)}\\
\end{tabular}
}
 % 插入 Stata 导出的表格

Based on the result, the production of tea is a realy great IV for us to identify the elasticity about the import price with import quantity which means the value is possible to be calculated by that.
What I want to explore is whether the Covid-19 influents the elasticity in the international trade.



\section{Conclusion}
The table above was directly imported from a separate file exported by Stata.

\end{document}